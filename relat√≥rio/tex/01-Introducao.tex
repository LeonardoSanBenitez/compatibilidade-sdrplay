%%%%%%%%%%%%%%%%%%%%%%%%%%%%%%%%%%%%%%%%%%%%%%%%%%%%%%%%%%%%%%%%%%%
%%%%%%%%%%%%%%%%%%%%%%%%%%%%%%%%%%%%%%%%%%%%%%%%%%%%%%%%%%%%%%%%%%%
\chapter{Introdução}
%%%%%%%%%%%%%%%%%%%%%%%%%%%%%%%%%%%%%%%%%%%%%%%%%%%%%%%%%%%%%%%%%%%
%%%%%%%%%%%%%%%%%%%%%%%%%%%%%%%%%%%%%%%%%%%%%%%%%%%%%%%%%%%%%%%%%%%
Um dos equipamentos mais importantes para a análise de Compatibilidade Eletromagnética é o analisador de espectro, capaz de - entre outros - medir a densidade de potência espectral (\textit{power spectral density} - PSD) de um sinal \cite{aulaSpirroEMC}. Tal equipamento possui alto custo, porém pode ser substituido por rádios definidos por software (\textit{software defined radio} - SDR) \cite{tccIgor}.

Este relatório descreve o processo de calibração e ajuste dos valores de amplitude de um SDR RSP1, onde considerou-se como "valor real"\  as medições realizadas com o analisador de espectro Rohde \& Schwarz HMS-X. O trabalho foi realizado como parte da disciplina de Tópicos Avançados em Compatibilidade Eletromagnética do curso de Engenharia Eletrônica do Instituto Federal de Educação, Ciência e Tecnologia de Santa Catarina.
