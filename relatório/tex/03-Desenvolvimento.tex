%%%%%%%%%%%%%%%%%%%%%%%%%%%%%%%%%%%%%%%%%%%%%%%%%%%%%%%%%%%%%%%%%%%
%%%%%%%%%%%%%%%%%%%%%%%%%%%%%%%%%%%%%%%%%%%%%%%%%%%%%%%%%%%%%%%%%%%
\chapter{Desenvolvimento}

%%%%%%%%%%%%%%%%%%%%%%%%%%%%%%%%%%%
\section{Revisão de Literatura}

\subsection{Calibração}
Estabelece o erro de medição e a incerteza de medição associada de
um instrumento, ao compará-lo a um padrão \cite{slidesCalibracaoLacen}.


Segundo \citeonline{VIM}, uma calibração pode ser expressa por meio de uma declaração, uma função de calibração, um diagrama de calibração, uma curva de calibração ou uma tabela de calibração.

\citeonline{vidotto2022} aponta que o processo de calibração de SDRs pode ser feito tanto para recepção quanto emissão, sendo que ambos podem ser realizados pela comparação com um equipamento de referência.


\subsection{Ajuste}
Operação destinada a levar um instrumento de medição a um funcionamento adequado à sua utilização \cite{slidesCalibracaoPaulo}. Após o ajuste físico ou manutenção de um instrumento ou sistema de medição, tal instrumento ou sistema de medição deve ser calibrado novamente \cite{slidesCalibracaoLacen}.

Diversos tipos de ajuste de um sistema de medição incluem o ajuste de zero, o
ajuste de defasagem (às vezes chamado ajuste de "\textit{offset}") e o ajuste de amplitude (às vezes chamada ajuste de ganho) \cite{VIM}.


\citeonline{VIM} ressalta a importância de não confundir a calibração com o ajuste de um sistema de medição, frequentemente denominado de maneira imprópria de "auto-calibração": O processo de ajuste elimina, total o parcialmente, o erro, mas a medição ainda possui incerteza.


\subsection{Erro}
Estabelece o quanto o resultado da medição de um instrumento se desviou
do valor nominal \cite{slidesCalibracaoLacen}.

Segundo \citeonline{VIM}, o conceito de erro de medição pode ser utilizado quando:
\begin{enumerate}
\item existe um único valor de referência, o que ocorre se uma calibração for realizada por meio de um padrão de medição com um valor medido cuja incerteza de medição é desprezável, ou se um valor convencional for fornecido; nestes casos, o erro de medição é conhecido;
\item se suponha que um mensurando é representado por um único valor verdadeiro ou um conjunto de valores verdadeiros de amplitude desprezável; neste caso, o erro de medição é desconhecido.
\end{enumerate}

\subsection{Incerteza}
Indica a faixa em que o "valor real"\ (valor verdadeiro convencional) pode estar \cite{slidesCalibracaoLacen}. É necessariamente um valor real não-negativo, representado pelo simbolo $u$, e caracteriza a dispersão dos valores atribuídos a um
mensurando \cite{VIM}.

Há duas abordagens na estimativa da incerteza de medição $u$:
tipo A e tipo B. A incerteza do tipo A é estimada a partir da
distribuição estatística dos valores \cite{slidesCalibracaoPaulo}.

$$
u_{tipo-A} = \text{desvio padrão} / sqrt(N)
$$

$$
\text{desvio padrão} = \sqrt{\frac{\sum(x_i - \bar{x})^2}{N-1}}
$$

$$
\text{N = Número de amostras}
$$

% se eu medir 10V e u=1, eu posso escrever a medida como sendo 10+-1, o que significa que há 68% de probabilidade do valor real estar entre 9 e 11 (o 68% vêm do 1 descrio padrão)

A incerteza do tipo B não é calculada de forma estatística, mas sim em especificações de fabricantes, em informações de publicações científicas, entre outros \cite{slidesCalibracaoPaulo}. Formas usuais de quantificar as incertezas do tipo B são: metade da menor divisão de escala (para equipamentos analógicos), resolução (para equipamentos digitais), valores publicados por autoridade competente, entre outros \cite{VIM}.



Incertezas-padrão, sejam elas do tipo A ou do tipo B, podem ser combinadas através de soma quadrática \cite{slidesCalibracaoPaulo}.

$$
u_c = \sqrt{u_1^2 + u_2^2 + ...}
$$

%%%%%%%%%%%%%%%%%%%%%%%%%%%%%%%%%%%
\section{Metodologia}
Montou-se um ambiente com o gerador de funções AFG3021B da Tektronix, uma placa de circuito impresso, uma antena de campo magnético, e um analisador de espectro (hora o SDR RSP1, hora o Rohde \& Schwarz HMS-X). Para ambos os analisadores utilizou-se os mesmos cabos, porém foi necessário utilizar um adaptador miniSMA-para-SMA para o Rohde \& Schwarz).

%Lalal (Figura \ref{fig:antena}). 

\begin{figure}[H]
    \centering
    \caption{Setup para as medições com o Rohde \& Schwarz HMS-X}
    \includegraphics[width=0.4\linewidth]{Images/setup-analisador.jpeg}
    \label{fig:analisador}
\end{figure}

\begin{figure}[H]
    \centering
    \caption{Setup para as medições com o SDR RSP1}
    \includegraphics[width=0.4\linewidth]{Images/setup-sdr.jpeg}
    \label{fig:sdr}
\end{figure}

\begin{figure}[H]
    \centering
    \caption{Antena e posicionamento utilizados em todas as medições}
    \includegraphics[width=0.4\linewidth]{Images/antena.jpeg}
    \label{fig:antena}
\end{figure}

Seguindo as instruções fornecidas em \cite{tccIgor}, o SDR RSP1 foi configurado da seguinte forma:
\begin{itemize}
    \item Ganho 1;
    \item Taxa de amostragem 1MHz;
    \item Tempo de aquisição 0,2s;
    \item Ao configurar o equipamento para medir uma determinada frequência (por exemplo, 10MHz), ajustou-se a frequência central de medição para que \textit{não} coincidisse com a frequência desejada (por exemplo, ajustando-a para 10.2MHz), eliminando assim a influência do nível DC na medição do SDR;
    \item Ao realizar a medição considerou-se apenas o valor máximo de amplitude dentro da janela de aquisição.
\end{itemize}

O analisador de espectro Rohde \& Schwarz HMS-X foi configurado da seguinte forma:

\begin{itemize}
    \item \textit{Receiver Mode} (medição de apenas uma frequência, em oposição ao \textit{Sweep Mode});
    \item Tempo de aquisição 0,2s;
    \item \textit{Peak detector};
    \item \textit{Step} de 1MHz.
\end{itemize}

Objetivando mensurar o erro e a incerteza do tipo A, realizou-se dois conjuntos de medições: 
\begin{enumerate}
    \item Medições de 10MHz a 200MHz espaçadas 10MHz entre si, para mensurar o erro;
    \item Medições de 10Mhz a 200Mhz espaçadas 50MHz entre si, repetindo cada medição 10 vezes, para mensurar a incerteza.
\end{enumerate}

Após essas medições, determinou-se um fator de ajuste, nomeado alfa, que deve ser multiplicado com a medição do SDR para obter-se o valor real da medição. Da mesma forma, obteve-se o fator $u$ que quatifica a incerteza padrão da medição.





\section{Apresentação dos Resultados}
\subsection{Conjunto de medições 1}

Gerou-se uma onda quadrada com amplitude de 8V e segui-se o procedimento apresentado na Metodologia. Como só existem as harmônicas pares, uma a cada duas medições era apenas o nível de ruído de fundo sendo medido. Calculou-se separadamente o fator de correção alfa para as amostras onde havia uma sinal presente (10Mhz, 30MHz, etc) e quando não havia um sinal presente (20MHz, 40MHz, etc).
\begin{table}[H]
    \centering
    \begin{tabular}{|l|l|l|l|l|}
    \hline
        Freq. (MHz) & Pico R\&S (dBm) & Pico SDR (dBm) & Alfa sinal & Alfa fundo \\ \hline
        10 & -34.08 & -29.93 & 1.14 & ~ \\ \hline
        20 & -71.75 & -68.89 & ~ & 1.04 \\ \hline
        30 & -42.25 & -40.55 & 1.04 & ~ \\ \hline
        40 & -69.81 & -68.70 & ~ & 1.02 \\ \hline
        50 & -45.84 & -45.74 & 1.00 & ~ \\ \hline
        60 & -69.73 & -62.71 & ~ & 1.11 \\ \hline
        70 & -56.76 & -37.63 & 1.51 & ~ \\ \hline
        80 & -71.53 & -64.51 & ~ & 1.11 \\ \hline
        90 & -68.01 & -57.81 & 1.18 & ~ \\ \hline
        100 & -71.84 & -68.83 & ~ & 1.04 \\ \hline
        110 & -71.73 & -68.65 & 1.04 & ~ \\ \hline
        120 & -72.16 & -65.54 & ~ & 1.10 \\ \hline
        130 & -71.98 & -59.61 & 1.21 & ~ \\ \hline
        140 & -72.23 & -64.09 & ~ & 1.13 \\ \hline
        150 & -72.87 & -59.57 & 1.22 & ~ \\ \hline
        160 & -73.09 & -68.63 & ~ & 1.07 \\ \hline
        170 & -72.44 & -68.93 & 1.05 & ~ \\ \hline
        180 & -71.22 & -68.69 & ~ & 1.04 \\ \hline
        190 & -70.87 & -68.98 & 1.03 & ~ \\ \hline
        200 & -71.49 & -69.13 & ~ & 1.03 \\ \hline
    \end{tabular}
\end{table}

%TODO: ver com o spirro se esse fator é aditivo ou multiplicativo

Como as harmônicas de uma onda quadrada possuem valor decrescente com a frequência, é esperado que os valores medidos para as frequências com harmônica não-nula sejam decrescentes. Observa-se também que as medições dos equipamentos se torna mais próxima com o aumento da frequência, convergindo para um valor de aproximadamente -70dB de ruído de fundo.

\subsection{Conjunto de medições 2}
Gerou-se uma onda quadrada com amplitude de 10V e segui-se o procedimento apresentado na Metodologia.

\begin{table}[H]
    \centering
    \begin{tabular}{|l|l|l|l|}
    \hline
        Freq (MHz) & u-tipo-A SDR & u-tipo-A R\&S & Fator de correção sinal médio \\ \hline
        10 & 0.02 & 0.005 & 1.14 \\ \hline
        50 & 0.04 & 0.02 & 1.01 \\ \hline
        100 & 0.08 & 0.2 & 1.04 \\ \hline
        150 & 0.04 & 0.2 & 1.21 \\ \hline
        200 & 0.08 & 0.2 & 1.04 \\ \hline
    \end{tabular}
\end{table}

Pode-se observar que ambos os equipamentos apresentam a incerteza positivamente correlacionada com a frequência de amostragem. 

Para as medições acima de 100 MHz o equipamento R\&S apresentou uma incerteza maior. É possível que tal se deva pela forma como as medições foram realizadas: enquanto as medições com o SDR foram realizadas com menos de um segundo de intervalo de tempo entre medições subsequêntes (devido ao procedimento de medição haver sido implementado em software), as medições com o R\&S foram realizadas manualmente e, portanto, houve um intervalo de alguns segundos entre uma medição e outra.