\chapter{Código de ajuste}

\begin{lstlisting}[language=Python]
import numpy as np
from typing import Tuple

adjust_table = [
  {
    'freq': 10,
    'alpha': 1.14,
    'u': 0.02,
  },
  {
    'freq': 30,
    'alpha': 1.04,
  },
  {
    'freq': 50,
    'alpha': 1.0,
    'u': 0.04,
  },
  {
    'freq': 70,
    'alpha': 1.51,
  },
  {
    'freq': 90,
    'alpha': 1.18,
  },
  {
    'freq': 100,
    'u': 0.08,
  },
  {
    'freq': 110,
    'alpha': 1.04,
  },
  {
    'freq': 130,
    'alpha': 1.21,
  },
  {
    'freq': 150,
    'alpha': 1.22,
    'u': 0.04,
  },
  {
    'freq': 170,
    'alpha': 1.05,
  },
  {
    'freq': 190,
    'alpha': 1.03,
  },
  {
    'freq': 200,
    'u': 0.08,
  },
]



def apply_adjust(frequency: float, measurement: float, adjust_table: list) -> Tuple[float, float]:
  '''
  The adjust_table is expected to be monotonically increasing in frequency
  Return the tuple (alpha, uncertainty)
  '''
  assert frequency >= adjust_table[0]['freq'], 'The device was not calibrated for this frequency range'
  assert frequency <= adjust_table[-1]['freq'], 'The device was not calibrated for this frequency range'
  alpha = np.interp(
      frequency, 
      [a['freq'] for a in adjust_table if 'alpha' in a],
      [a['alpha'] for a in adjust_table if 'alpha' in a],
  )
  uncertainty = np.interp(
      frequency, 
      [a['freq'] for a in adjust_table if 'u' in a], 
      [a['u'] for a in adjust_table if 'u' in a],
  )
  return (alpha, uncertainty)


frequency = 25
measurement = -70
alpha, uncertainty = apply_adjust(frequency, measurement, adjust_table)
print(f"At {frequency} MHz, we measured {measurement} V, which is adjusted to {measurement*alpha}+-{uncertainty} V\t\t(alpha={alpha})")
\end{lstlisting}